\section{Модель}\label{sec:model}

Модель представляет из себя двумерную кристаллическую решетку на плоском
торе. Каждый элемент решетки имеет свой спин (трехмерный единичный вектор).

Атомы на решетке имеют связь только с 4 своими ближайшими соседями,
то есть $i$ атом имеет связь с $i-1$, $i+1$, $i-x$, $i+x$ элементами,
если $x$ и $y$ задают размеры решетки, а узлы индексируются как
$index = p_x + x*p_y$, где $p_x$ и $p_y$ это положение
узла в решетке.

Между соседними атомами с индексами $i$ и $j$ установлена связь взаимодействия
Дзялошинского-Мория $\mathbf D_{i,j}$, в направлении от одного узла к другому.
Стоит отметить, что $\mathbf D_{i,j} = -D_{j,i}$.

Основные характеристики модели:
\begin{itemize}
\item $b$ -- вектор характеризующий магнитное поле, далее для
    усовершенствования можно заменить на векторное поле, чтобы смоделировать
    поведение в неоднородной среде
\item $i$ -- сила межатомного взаимодействия
\item $\mathrm{K_0}$ -- абсолютное значение анизотропии
\item $\mathbf K$ -- направление вектора анизотропии, единичный трехмерный вектор
\item $\lambda$ -- параметр диссипации (для среды без диссипации $\lambda=0$)
\item $\gamma$ -- феноменологическая постоянная из уравнения Ландау-Лифшица
\item $x$ и $y$ -- размер решетки
\end{itemize}

Для сравнения эффективности метода, помимо самого симплектического метода
Рунге-Кутта \ref{sec:symplectic-integrator}, был реализован метод Эйлера
\ref{sec:sol-euler} и неявный метод Рунге-Кутта 4го порядка.


\section{Используемые технологии}

Для реализации модели~\ref{sec:model} был выбран язык программирования
\emph{python}, в связи с простотой синтаксиса и большим выбором
высокопроизводительных библиотек для математического моделирования, и модуль
для научных вычислений \emph{scipy}~\cite{scipy}, являющийся одним из самых
эффективных и популярных на момент написания работы.  Приемуществом
предлагаемого алгоритма является отсутствие в вычислениях сложных операций (с
точки зрения производительности выполнения операций на ЭВМ), в программе
использованы только матричные функции сложения, умножения модуля научных
вычислений \emph{numpy}~\cite{numpy}.  Графики и рисунки приведенные в работе
построены с использованием модуля \emph{matplotlib}~\cite{matplotlib}.

Вычисления производились на тестовом стенде с характеристиками:
\begin{itemize}
    \item ЦПУ: AMD A8-7100 Radeon R5, 4 ядра, 1800\,МГц
    \item ОС: Arch Linux, x86\_64 Linux 4.5.4-1-ARCH
    \item ЗУПВ:
        \begin{itemize}
            \item SODIMM DDR3 1600\,МГц, 4\,Гб, RMT3170ME68F9F1600
            \item SODIMM DDR3 1600\,МГц, 8\,Гб, CT102464BF160B.M16
        \end{itemize}
    \item \emph{python} 2.7.11
    \item \emph{scipy} 0.17.1
    \item \emph{numpy} 1.11.0
    \item \emph{matplotlib} 1.5.1
\end{itemize}


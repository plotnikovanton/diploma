\section{Симплектические методы}
Симплектическими методами(симплектический интегратор)
называют численные схемы интегрирования Гамильтоновых систем.

Гамильтонова система описывает динамическую систему без диссипации и
определяется системой уравнений, которая может быть записана в форме уравнений
Гамильтона:
\begin{equation}
    \begin{cases}
        \dot p_i = -\dfrac{\partial \mathbf H}{\partial q_i}\\
        \dot q_i = \dfrac{\partial \mathbf H}{\partial p_i}
    \end{cases},
    \qquad i = 1\dots n
\end{equation}
где $\mathbf H$ -- функция Гамильтона, обычно имеющая смысл энергии. Координаты
$(p, q)$ называются каноническими.

Идея симплектического метода заключается в том, чтобы 

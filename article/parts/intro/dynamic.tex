\section{Уравнения Гамильтона в теоретической динамике}
Динамическую систему с $s$ степенями свободы
можно описать с помощью $2s$ обыкновенных дифференциальных
уравнений. Удобным способом записи является запись с помощью уравнений
Гамильтона
\begin{equation}\label{eq:ham-system}
    \begin{cases}
        \dot q_k = \dfrac{\partial H(p, q)}{\partial p_k}\\
        \dot p_k = -\dfrac{\partial H(p, q)}{\partial q_k}
    \end{cases}, \qquad k = 1\dots s
\end{equation}

Смыслом гамилтониана системы обычно является энергия. Рассмотрим такую систему,
где гамильтониан не зависит явно от времени, тогда
\begin{equation}
    \frac{\partial H}{\partial d} = 0,
\end{equation}
легко показать что в такой системе
гамильтониан, записанный в виде \ref{eq:ham-system}, не меняется со временем
\begin{multline}
    \frac{d H}{dt} = \sum_j\left(\frac{\partial H}{\partial q_j}
    \dot q_j + \frac{\partial H}{\partial p_j}\dot p_j\right) + \frac{\partial
    H}{\partial t}
    =\\=
    \sum_j\left(\frac{\partial H}{\partial q_j} \frac{\partial H}{\partial p_i}
    - \frac{\partial H}{\partial p_j}\frac{\partial H}{\partial q_i} \right)
    + \frac{\partial H}{\partial t} = \frac{\partial H}{\partial t} = 0
\end{multline}

Подробнее изучить материал можно, например, в работе \cite[стр.~123]{1974} и
\cite[стр. 260]{1980}

Такие системы обычно невозможно решить в символьном виде, поэтому нужно
прибегать к численным методам.

\section{Метод Ньютона}
При решении задачи коши с помощью методов Рунге-Кутта возникает надобность
решить нелинейное уравнение \ref{eq:rk-not-linear}.

Метод ньютона -- один из итерационных численных методов для отыскания корня.
Пусть дана система уравнений
\begin{equation}\label{eq:newton-needed-form}
    \begin{cases}
        f_1(x_1, x_2,\dots, x_n) = 0\\
        \dots\\
        f_1(x_1, x_2,\dots, x_n) = 0
    \end{cases}
\end{equation}
и есть начальное приближение $x^{0}$, тогда следующее приближение вычисляется
из системы линейных уравнений
\begin{equation}\label{eq:newton-generalized-iteration}
    f_i(x^j) + \sum_{k=1}^n \frac{\partial f_i}{\partial x_k}(x^j)
    (x_k^{j+1}-x_k^{j}) = 0,  i=1\dots n
\end{equation}

Критерием остановки метода может служить, например, $|x^i - x^{i-1}|<\epsilon$

Для решения системы \ref{eq:newton-needed-form} можно снова прибегнуть к помощи
численных методов, например к методу би-сопряженных градиентов.


\section{Уравнение Ландау-Лифшица}
В данной работе исследуется динамика системы, описанной уравнением
Ландау-Лифшица.

Уравнение Ландау-Лифшица-Гильберта описывает движение векторов намагниченности в
кристаллических решетках фери- и ферромагнетиков см. \cite{1984}.
\begin{equation}\label{eq:lan-lif-full}
    \frac{\partial \mathbf{S}}{\partial t} = - |\gamma|\left[\mathbf{S}\times
    \heff\right]-
    |\gamma|\lambda S \times \left( S \times \heff \right)
\end{equation}

Тут $\gamma$ некоторая феноменологическая постоянная,
$\heff$ -- эффективное магнитное поле, которое выражается через градиент
энергии $\mathbf E$ (вид гамильтониана см. \cite[стр. 2]{Hagemeister2015}):
\begin{equation}
    \heff = -\nabla \mathbf E \label{eq:heff-def}
\end{equation}


\section{Актуальность работы}

Последнее время часто поднимается тема магнитных скирмионов в научных работах и
публикациях.

Скирмионы -- это квазичастица, представляющая собой структуру, выстраивающуюся
из спинов нескольких атомов, обзор скирмионной системы на двумерной
кристаллической решетке можно посмотреть, например в \cite{Yu2010}. За счет
стабильности (подробнее можно прочесть в статье \cite{nucleation},
опубликованной в журнале Science) и своих малых размеров (порядка 1-2
нанометров) они представляют интерес в использовании в качестве ячеек магнитной
памяти. В недавно-опубликованной статье \cite{stat-ant-dyn-prop-of-skyr},
ученые из университета Тохоку изучили динамику поведения скирмионов в
антиферромагнетиках и предсказали их поведение с учетом силы Магнуса, таким
образом поведение магнитных скирмионов можно считать потенциально управляемым.

Такая система описывается уравнением Ландау-Лифшица, которое крайне сложно
проинтегрировать символьно, поэтому для исследования динамики систем
скирмионных структур будет полезно вывести метод, с помощью которого можно
моделировать поведение скирмионов в динамически меняющихся условиях, например
воздействие на них точечного заряда или изменении магнитного поля, который при
этом будет достаточно эффективен и точен для наблюдения динамики системы в
''реальном времени``. Также о такой системе известно что энергия (в
без диссипативной среде) должна сохранятся, поэтому нельзя пользоваться
обычными численными методами высоких порядков, таких как методы Рунге-Кутта,
поскольку они не гарантируют сохранений каких-либо инвариантов систем.


В настоящей работе предлагается использование симплектических методов
Рунге-Кутта для интегрирования уравнения Ландау-Лифшица.
В главе \ref{ch:survey} дан обзор используемых в решении
поставленной задачи численных методов, напоминается уравнение Гамильтона и его
свойства и уравнение Ландау-Лифшица.
В главе \ref{ch:problem} описывается конкретная постановка проблемы и выводятся
необходимые формулы для использования их в приведенных в главе \ref{ch:survey}
методах.
Далее в \ref{ch:realisation} приведены результаты проведенных экспериментов,
представлены сравнительные графики зависимостей исследуемых характеристик
моделей и сравнение эффективности работы предлагаемого алгоритма и классических
алгоритмов Эйлера и Рунге-Кутта.


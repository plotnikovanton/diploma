\section{Уравнение Ландау-Лифшица}

Уравнение Ландау-Лифшица в форме Ландау-Лифшица-Гильберта описывает движение
векторов намагниченности в кристаллических решетках фери- и ферромагнетиков в
системе с диссипацией подробнее см.~\cite{lan-lif-again}.
\begin{equation}\label{eq:lan-lif-full}
    \frac{\partial \mathbf{S}}{\partial t} = - |\gamma|\left[\mathbf{S}\times
    \heff\right]-
    |\gamma|\lambda S \times \left( \mathbf S \times \heff \right),
\end{equation}
тут $\gamma$ некоторая феноменологическая постоянная,
$\heff$ -- эффективное магнитное поле, которое выражается через градиент
энергии $\mathbf E$ (вид гамильтониана см.~\cite[с. 2]{Hagemeister2015}):
\begin{equation}
    \heff = -\nabla \mathbf E \label{eq:heff-def}.
\end{equation}


\section{Уравнения Гамильтона в электродинамике}
Динамическую систему с $s$ степенями свободы можно описать с помощью $2s$
обыкновенных дифференциальных уравнений. Удобным способом записи является
запись с помощью уравнений Гамильтона
\begin{equation}\label{eq:ham-system}
    \begin{dcases}
        \dot q_k = \frac{\partial H(p, q)}{\partial p_k}\\
        \dot p_k = -\frac{\partial H(p, q)}{\partial q_k}
    \end{dcases}, \qquad k = 1\dots s.
\end{equation}

В простых случаях гамильтониан представляет собой энергию системы. Рассмотрим
автономную систему, т.е. систему гамильтониан не зависит явно от времени, тогда
\begin{equation}
    \frac{\partial H}{\partial t} = 0,
\end{equation}
легко показать что в такой системе функция Гамильтона, записанная в виде
\ref{eq:ham-system}, не меняется со временем
\begin{multline}
    \frac{d H}{dt} = \sum_j\left(\frac{\partial H}{\partial q_j}
    \dot q_j + \frac{\partial H}{\partial p_j}\dot p_j\right) + \frac{\partial
    H}{\partial t}
    =\\=
    \sum_j\left(\frac{\partial H}{\partial q_j} \frac{\partial H}{\partial p_i}
    - \frac{\partial H}{\partial p_j}\frac{\partial H}{\partial q_i} \right)
    + \frac{\partial H}{\partial t} = \frac{\partial H}{\partial t} = 0.
\end{multline}

Подробнее изучить материал можно, например, в работах
\cite[с.~123]{hamilton-mech} и \cite[с.~260]{classic-mech}.

Такие системы обычно невозможно решить в символьном виде, поэтому нужно
воспользоваться какими-нибудь численными методами.

\section{Метод Рунге-Кутта}
Также широко популярен класс численных методов решения задачи коши именуемыми
методами Рунге-Кутта.

Пусть есть задача Коши (\ref{eq:koshi}), тогда
в общем виде итерационная схема неявного метода Рунге-Кутта имеет вид:
\begin{figure}[h]
    \renewcommand{\arraystretch}{1.2}
    \centering
    \begin{tabular}{c|ccc}
        $c_1$    & $a_{11}$ & $\ldots$ & $a_{1s}$ \\
        $\vdots$ & $\vdots$ & $\ddots$ & $\vdots$ \\
        $c_s$    & $a_{11}$ & $\ldots$ & $a_{ss}$ \\ \hline
                 & $b_{1}$  & $\ldots$ & $b_{s}$ \\
    \end{tabular}
    \caption{Таблица для общего вида}
\label{tab:tableau_basic}
\end{figure}

\begin{equation}\label{eq:rk-schema}
    y_{n+1} = y_n + h\sum_{j=1}^s b_j f(x_n+c_jh, \xi_j),
\end{equation}
где $\xi$ вычисляется из нелинейного уравнения
\begin{equation}\label{eq:rk-not-linear}
    \xi_j = y_n + h\sum_{i=1}^{s}a_{ji} f(x_n+c_jh, \xi_i).
\end{equation}

Коэффициенты $c_i$, $a_ij$ и $b_i$ вычисляются из разложения функций $y_n,
f$ в ряд Тейлора и приравнивания коэффициентов при степенях $h^{(p-1)}$ к нулю,
где $p$ -- порядок степени аппроксимации схемы. Подробнее
см.~\cite[с.~75]{chilsl-metodi}.

Пусть $M=\left( m_{ij} \right)^s_{i,j=1}$ -- матрица вещественных чисел размера
$s\times s$, где
\begin{equation}
    m_{ij} = b_i a_{ij} + b_j a_{ji} - b_i b_j,\qquad i,j=1,\dots,s,
\end{equation}
тогда справедлива следующая теорема
\begin{theorem}
    Если $M = 0$, тогда метод Рунге-Кутта является симплектическим.
\end{theorem}

В данной работе рассматривается симплектический метод Рунге-Кутта
второго порядка, имеющий четвертый порядок точности,
он же метод Гаусса-Лежандра-Рунге-Кутта. Для него
таблица \ref{tab:tableau_basic} выглядит как таблица \ref{tab:gauss-lagr}.

\begin{figure}[h]
    \renewcommand{\arraystretch}{1.8}
    \centering
    \begin{tabular}{c|cc}
        $\frac12 - \frac{\sqrt3}6$ & $\frac14$                  & $\frac14 - \frac{\sqrt3}6$ \\
        $\frac12 + \frac{\sqrt3}6$ & $\frac14 + \frac{\sqrt3}6$ & $\frac14$ \\ \hline
                                   & $\frac12$                  & $\frac12$
    \end{tabular}
    \caption{Таблица для метода Гаусса-Лагранжа-Рунге-Кутта}
\label{tab:gauss-lagr}
\end{figure}


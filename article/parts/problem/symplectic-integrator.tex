\section{Симплeктический интегратор}\label{sec:symplectic-integrator}
Чтобы воспользоваться симплектическим методом нужно убедиться в том,
что энергия в системе, описанной уравнением Ландау-Лифшица \ref{eq:lan-lif},
действительно должна сохраняться. Для этого посмотрим на ее дифференциал и
убедимся что он равен $0$.
\begin{equation}
    \frac{d\mathbf E(S(t))}{dt} =
    \left<\nabla_S \mathbf E, \dot{\mathbf S} \right> =
    \left< \nabla_\mathbf S \mathbf E, \gamma \mathbf S \times \nabla_S \mathbf E
    \right> = 0
\end{equation}

Так же Гамильтониан должен быть представлен в симплектической форме:
\begin{equation}\label{eq:ham_sym_form}
    \begin{cases}
        \dot q^{[n]} = \dfrac{\partial \mathbf E}{\partial p^{[n]}}
        \\
        \dot p^{[n]} = - \dfrac{\partial \mathbf E}{\partial q^{[n]}}
    \end{cases}
\end{equation}

Поскольку $\mathbf S^{[n]}$ всегда единичный вектор в $\mathds R^3$
,в определенной фиксированной точке, то его состояние можно однозначно
определить парой координат в ортогональном базисе, например в сферических
координатах или с помощью длин ортогональных векторов на сфере.
Таким образом можно ввести новый базис для
каждого атома в решетке и представить $\mathbf S^{[n]}$
следующим образом:
\begin{equation}
    \mathbf S^{[n]} = \mathbf S^{[n]}(q^{[n]}, p^{[n]})
\end{equation}

Итак необходимо убедиться в эквивалентности:
\begin{equation}\label{eq:is_ham_split_eq}
    \mathbf S^{[n]} = |\gamma| \mathbf S^{[n]}\times \nabla_{\mathbf S^{[n]}}
    \mathbf E
    \overset{?}{\Leftrightarrow}
    \begin{cases}
        \dot q^{[n]} = \frac{\partial \mathbf E}{\partial p^{[n]}}
        \\
        \dot p^{[n]} = -\frac{\partial \mathbf E}{\partial p^{[n]}}
    \end{cases}
\end{equation}

Для удобства расчетов далее пологается $\gamma=1$. Если есть необходимость
задать эту консанту отличной от $1$, то можно переопределинь энергию $\mathbf
E$ так, чтобы внести в нее эту поправку, тогда будут верны все ниже указанные
расчеты.

Посчитаем производную по времени у Гамильтониана в симплектической
форме
\begin{multline}
    \dot{\mathbf S}^{[n]} = \frac{d}{dt}\mathbf S^{[n]}(q^{[n]}(t),p^{[n]}(t))=
    \frac{\partial \mathbf S^{[n]}}{\partial q^{[n]}} \cdot \dot q^{[n]} +
    \frac{\partial \mathbf S^{[n]}}{\partial q^{[n]}}
    =\\=
    \frac{\partial \mathbf S^{[n]}}{\partial q^{[n]}}
    \frac{\partial \mathbf E}{\partial p^{[n]}}
    -
    \frac{\partial \mathbf S^{[n]}}{\partial p^{[n]}}
    \frac{\partial \mathbf E}{\partial q^{[n]}}
    =
    \left|\begin{matrix}
        \Sn & \frac{\partial \Sn}{\partial q^{[n]}} &
        \frac{\partial \Sn}{\partial p^{[n]}}
        \\
        1 & 0 & 0
        \\
        \frac{\partial \mathbf E}{\partial S} &
        \frac{\partial \mathbf E}{\partial q^{[n]}} &
        \frac{\partial \mathbf E}{\partial p^{[n]}}
    \end{matrix}\right| = \Sn \times \nabla_{\Sn}\mathbf E
\end{multline}
что и требовалось доказать в \ref{eq:is_ham_split_eq}.

Таким образом можно воспользоваться симплектическим методом Рунге-Кутта,
обзор которого можно посмотреть, например, в статье \citet{Markiewicz1999},
и пользоваться приводимыми там
формулами без явной записи гамильтониана в виде \ref{eq:ham_sym_form}.


Так как исследуемая система автономна, то в уравнениях \ref{eq:rk-schema},
\ref{eq:rk-not-linear}
функция $f(t_n + c_jh, \xi_j)$ принимает вид $f(\xi_j)$

Для исследуемой модели $f(x)$ есть правая часть уравнения Ландау-Лифшица
\ref{eq:lan-lif}:
\begin{equation}
    f(\xi) = -\mathbf S \times \heff(\xi) = \mathbf S \times \nabla_{\mathbf S}
    \mathbf E
\end{equation}

Тогда итерационная схема \ref{eq:rk-schema} для исследуемой
модели будет записана в виде:
\begin{gather}
    \mathbf S^{[n]}_{k+1} = \Sn_k + h\sum_{j=1}^s b_j \left[\Sn_k \times
    \nabla_{\xi^{[n],k}_j} \mathbf E^k \right]
    \\
    \xi_j^{[n], k} = \Sn_k + h\sum_{i=1}^s a_{j,i} \left[ -\xi_i^{[n],k}
    \times \nabla_{\xi^{[n],k}_i} \mathbf E^k \right]\label{eq:rk-not-linear}
\end{gather}
где
\begin{align}
    k &\text{ -- номер шага в методе Рунге-Кутта} \\
    n &\text{ -- номер узла в решетке}
\end{align}

Для вычисления каждого следующего состояния системы необходимо решить
нелинейное уравнение \ref{eq:rk-not-linear}. Для этого можно воспользоваться,
каким-нибудь численным методом, например методом Ньютона.


\section{Симплeктический интегратор}\label{sec:symplectic-integrator}
Чтобы воспользоваться симплектическим методом нужно убедиться в том,
что энергия в системе, описанной уравнением Ландау-Лифшица (\ref{eq:lan-lif}),
действительно должна сохраняться. Для этого посмотрим на ее дифференциал и
убедимся что он равен $0$:
\begin{equation}
    \frac{d\mathbf E(S(t))}{dt} =
    \Braket{\nabla_S \mathbf E | \dot{\mathbf S} } =
    \Braket{\nabla_\mathbf S \mathbf E| \gamma \mathbf S \times \nabla_S
    \mathbf E
    } = 0,
\end{equation}
также гамильтониан должен быть представлен в симплектической форме:
\begin{equation}\label{eq:ham_sym_form}
    \begin{cases}
        \dot q^{[n]} = \dfrac{\partial \mathbf E}{\partial p^{[n]}}
        \\
        \dot p^{[n]} = - \dfrac{\partial \mathbf E}{\partial q^{[n]}}
    \end{cases}\quad.
\end{equation}

Поскольку $\mathbf S^{[n]}$ всегда единичный вектор в $\mathds R^3$, его
состояние можно однозначно определить парой координат в ортогональном базисе,
например в сферических координатах или с помощью длин ортогональных векторов на
сфере. Таким образом можно ввести новый базис для каждого атома в решетке и
представить $\mathbf S^{[n]}$
следующим образом:
\begin{equation}
    \mathbf S^{[n]} = \mathbf S^{[n]}(q^{[n]}, p^{[n]}).
\end{equation}

Итак необходимо убедиться в эквивалентности:
\begin{equation}\label{eq:is_ham_split_eq}
    \mathbf S^{[n]} = |\gamma| \mathbf S^{[n]}\times \nabla_{\mathbf S^{[n]}}
    \mathbf E
    \overset{?}{\Leftrightarrow}
    \begin{cases}
        \dot q^{[n]} = \frac{\partial \mathbf E}{\partial p^{[n]}}
        \\
        \dot p^{[n]} = -\frac{\partial \mathbf E}{\partial p^{[n]}}
    \end{cases}.
\end{equation}

Для удобства расчетов далее пологается $\gamma=1$. Если есть необходимость
задать эту консанту отличной от $1$, то можно переопределинь энергию $\mathbf
E$ так, чтобы внести в нее эту поправку, тогда будут верны все ниже указанные
расчеты.

Посчитаем производную по времени у гамильтониана в симплектической
форме
\begin{multline}\label{eq:ham-deriv-matrix}
    \dot{\mathbf S}^{[n]} = \frac{d}{dt}\mathbf S^{[n]}(q^{[n]}(t),p^{[n]}(t))=
    \frac{\partial \mathbf S^{[n]}}{\partial q^{[n]}} \cdot \dot q^{[n]} +
    \frac{\partial \mathbf S^{[n]}}{\partial q^{[n]}}
    =\\=
    \frac{\partial \mathbf S^{[n]}}{\partial q^{[n]}}
    \frac{\partial \mathbf E}{\partial p^{[n]}}
    -
    \frac{\partial \mathbf S^{[n]}}{\partial p^{[n]}}
    \frac{\partial \mathbf E}{\partial q^{[n]}}
    =
    \left|\begin{matrix}
        \Sn & \frac{\partial \Sn}{\partial q^{[n]}} &
        \frac{\partial \Sn}{\partial p^{[n]}}
        \\
        1 & 0 & 0
        \\
        \frac{\partial \mathbf E}{\partial S} &
        \frac{\partial \mathbf E}{\partial q^{[n]}} &
        \frac{\partial \mathbf E}{\partial p^{[n]}}
    \end{matrix}\right|.
\end{multline}
Здесь, полученная в итоге, матрица представляет векторное произведение в базисе
записанном первой строкой, вторая строка есть $\mathbf S^{[n]}$ и 3 представляет
градиент $\mathbf E$ по $\mathbf S^{[n]}$, таким образом
\begin{equation}
    \dot{\mathbf S}^{[n]} = \Sn \times \nabla_{\Sn}\mathbf E,
\end{equation}
что и требовалось доказать в (\ref{eq:is_ham_split_eq}).

Таким образом можно воспользоваться симплектическим методом Рунге-Кутта,
обзор которого можно посмотреть, например, в статье
\cite{survey-on-symplectic-integrators},
и пользоваться приводимыми там
формулами без явной записи гамильтониана в виде (\ref{eq:ham_sym_form}).


Так как исследуемая система автономна, то в уравнениях (\ref{eq:rk-schema}),
(\ref{eq:rk-not-linear})
функция $f(t_n + c_jh, \xi_j)$ принимает вид $f(\xi_j)$.
Для исследуемой модели $f(x)$ есть правая часть уравнения Ландау-Лифшица
(\ref{eq:lan-lif}):
\begin{equation}
    f(\xi) = -\mathbf S \times \heff(\xi) = \mathbf S \times \nabla_{\mathbf S}
    \mathbf E(\xi).
\end{equation}

Тогда итерационная схема (\ref{eq:rk-schema}) для исследуемой
модели будет записана в виде:
\begin{gather}
    \mathbf S^{[n]}_{k+1} = \Sn_k + h\sum_{j=1}^s b_j \left[\xi^{[n]}_k \times
    \nabla_{\xi^{[n],k}_j} \mathbf E^k \right],
    \\
    \xi_j^{[n], k} = \Sn_k + h\sum_{i=1}^s a_{j,i} \left[ -\xi_i^{[n],k}
    \times \nabla_{\xi^{[n],k}_i} \mathbf E^k \right]\label{eq:rk-not-linear},
\end{gather}
где
\begin{itemize}
    \item $k$ -- номер шага в методе Рунге-Кутта,
    \item $n$ -- номер узла в решетке.
\end{itemize}

Для вычисления каждого следующего состояния системы необходимо решить
нелинейное уравнение (\ref{eq:rk-not-linear}). Для этого можно воспользоваться,
каким-нибудь численным методом, например методом Ньютона.

Отметим, что выше мы не переходим к локальным двумерным координатам, а остаемся
в трехмерном пространстве, то необходимо учесть, чтобы длины векторов,
характеризующие направление спинов атомов должны сохраняться. Запишем этот
инвариант в эквивалентном виде следующем виде:
\begin{equation}
    L^{[n]} = \Braket{\Sn | \Sn}.
\end{equation}
Заметим, что $L$ является квадратичной формой, тогда согласно теореме
\ref{th:quadratic-integrals} (подробнее см. \cite{rk-schemes-for-ham-sys})
симпелектический метод Рунге-Кутта будет сохранять
этот инвариант.
\begin{theorem}\label{th:quadratic-integrals}
    Симплектический метод Рунге-Кутта сохраняет все инварианты в квадратичной
    форме Гамильтоновой системы.
\end{theorem}

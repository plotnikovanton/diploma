\section{Уравнение Ландау-Лифшица}
Рассмотрим систему описываемую уравнением Ландау-Лифшица \ref{eq:lan-lif-full}.
Второе слагаемое в этом уравнении является диссипативным членом.
В данной работе исследуются симпликтические методы интегрирования уравнения
Ландау-Лифшица с целью сохранения полной энергии системы, поэтому далее
уравнение \ref{eq:lan-lif-full} будет рассматриваться как уравнение для
бездиссипативной среды и выглядеть следующим образом:
\begin{equation}\label{eq:lan-lif}
	\frac{\partial \mathbf S}{\partial t} = - |\gamma|[\mathbf S\times \heff]
\end{equation}

Возьмем энергию вклад в которую вносят:
\begin{itemize}
    \item $\mathbf B$ -- внешнее магнитное поле
    \item $\mathrm{K0}, \mathbf K$ -- анизотропия, где $\mathrm{K0}$ и $\mathbf
        K$ линейный вклад в анизотропию и единичный вектор направления вектора
        анизотропии
    \item $J^{[n,m]}$ -- сила межатомного взаимодействия между атомами $n$ и
        $m$
    \item $D^{[n,m]}$ -- сила Дзялошинского-Мория между атомами $n$ и $m$
\end{itemize}
\begin{multline}
    \mathbf E = -\sum_n \mathbf B \cdot \Sn - \mathrm{K0} \sum_n |\mathbf K \cdot
    \Sn |^2
    -\\-
    \sum_{<n,m>} J^{[n,m]} \Sn - \sum_{<n,m>} \mathbf D^{[n,m]}\cdot
    (\Sn \times \mathbf S^{[m]})
\end{multline}

Для удобства дальнейших расчетов перепишем в эффективное магнитное поле
\ref{eq:heff-def} в виде
\begin{equation}
    \heff^{[n]} = \nabla_{S^{[n]}}\mathbf{E} = A\mathbf{S}^{[n]} + \mathbf
    B^{[n]},\label{eq:heff}
\end{equation}
где
\begin{equation}
    A\textbf S^{[n]} = -2\mathrm{K0} \mathbf{K} (\mathbf{K}\cdot \mathbf{S}^{[n]}) -
    \sum_m \mathbf J^{[n, m]} \mathbf S^{[m]} - \sum_m \mathbf S^{[m]} \times
    \mathbf{D}^{[n,m]}
\end{equation}


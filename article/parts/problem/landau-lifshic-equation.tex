\section{Уравнение Ландау-Лифшица}
Рассмотрим систему описываемую уравнением Ландау-Лифшица
(\ref{eq:lan-lif-full}).
Второе слагаемое в этом уравнении является диссипативным членом.
В данной работе исследуются симпликтические методы интегрирования уравнения
Ландау-Лифшица с целью сохранения полной энергии системы, поэтому далее
уравнение (\ref{eq:lan-lif-full}) будет рассматриваться как уравнение для
бездиссипативной среды и выглядеть следующим образом:
\begin{equation}\label{eq:lan-lif}
	\frac{d \mathbf S}{d t} = - |\gamma|[\mathbf S\times \heff].
\end{equation}

Будем рассматривать систему вклад в энергию которой вносят:
\begin{itemize}
    \item $\sum\limits_n \Braket{\mathbf B | \Sn}$ -- внешнее магнитное поле
    \item $\mathrm{K_0} \sum\limits_n \left| \Braket{\mathbf K | \Sn }
        \right|^2 $
        -- анизотропия, где $\mathrm{K_0}$ -- линейный вклад в анизотропию и
        $\mathbf K$ -- единичный вектор направления вектора анизотропии
    \item $\sum\limits_{n\thicksim m} J^{[n,m]}$ -- межатомное взаимодействие
        между атомами $n$ и $m$
    \item $\Braket{\mathbf D^{[n,m]} | \left(\Sn \times
        \mathbf S^{[m]}\right) }$ -- вектор Дзялошинского-Мория между атомами
        $n$ и $m$
\end{itemize}
\begin{multline}
    \mathbf E = -\sum_n \Braket{\mathbf B | \Sn} - \mathrm{K_0} \sum_n
    \left| \Braket{\mathbf K | \Sn } \right|^2
    -\\-
    \sum_{n\thicksim m} J^{[n,m]} \Sn - \sum_{n\thicksim m}
    \Braket{\mathbf D^{[n,m]} | \left(\Sn \times \mathbf S^{[m]}\right)}.
\end{multline}

Для удобства дальнейших расчетов перепишем в эффективное магнитное поле
(\ref{eq:heff-def}) в виде
\begin{equation}
    \heff^{[n]} = \nabla_{\mathbf S^{[n]}}\mathbf{E}
\end{equation}
где
\begin{multline}
    \nabla_{\mathbf S^{[n]}}\mathbf E
    =\\=
    \nabla_{\mathbf S^{[n]}} \left({%
    -\sum_n \Braket{\mathbf S^{[n]} | \mathbf K \mathrm{K_0}\Braket{\mathbf K |
    \mathbf S^{[n]}}} - \frac12 \Braket{\sum_n \mathbf S^{[n]} |
    \sum_{n\thicksim m} J^{[n,m]}\mathbf S^{[m]}}
    }\right.-\\-\left.{%
    \frac12\Braket{\mathbf S^{[n]} | \sum_{n\thicksim m} \mathbf S^{[n]} \times
    \mathbf D^{[n,m]}}
    -
    \sum_n \Braket{\mathbf B | \mathbf S^{[n]}}
    }\right)
    =\\=
    \underbrace{%
    -2\mathrm{K_0}\mathbf K \Braket{\mathbf K | \mathbf S^{[n]}}
    -
    \sum_{n\thicksim m}J^{[n,m]}\mathbf S^{[m]} - \sum_{n\thicksim m}
    \mathbf S^{[m]}\times \mathbf D^{[n,m]}
    }_{A\mathbf S^{[n]}}
    -
    \mathbf B.
\end{multline}
Для удобства дифференцирования далее будем рассматривать уравнение
(\ref{eq:heff-def}) в виде
\begin{equation}
    \heff^{[n]}=\nabla_{\mathbf S^{[n]}} = A\mathbf S^{[n]} - \mathbf B.
\end{equation}

\section{Метод Ньютона}
Методом Ньютона в обобщенном виде удобно искать численное решение подобной
системы:
\begin{equation}\label{eq:newton_base}
    \begin{cases}
        f_1(x_1, x_2, \dots, x_n) = 0\\
        \dots\\
        f_n(x_1, x_2, \dots, x_n) = 0
    \end{cases}
\end{equation}

Выбрав некоторое начальное приближение $\bar x^{[0]}$ следующие приближения находятся
из решения системы уравнений:
\begin{equation}\label{eq:newton_generalized_iteration}
    f_i + \sum_{k=1}^n \frac{\partial f_i}{\partial x_k}
    (x_k^{[j+1]}-x_k^{[j]})
\end{equation}

\begin{remark}
    Для решения системы можно воспользоваться методом би-сопряженных градиентов
    \ref{REF TO BI CONJ}.
\end{remark}

Для того чтобы воспользоваться методом ньютона необходимо выделить
многомерную функцию $f$ и вычислить ее производную. Перепишем
уравнение~\ref{eq:rk_xi_equation} в виде:
\begin{equation}
    S_n + h\sum_{i=1}^s a_{ji}\cdot \left(
        -\gamma S_n \times H^{eff}_{i} - \gamma\lambda S_n \times
    \left(S_n \times H^{eff}_{i}\right)\right) - S_j = 0
    \label{eq:equation_in_newton_form}
\end{equation}

Вычислим $\nabla_{S_n} f$:
\begin{multline*}
    \nabla_{S_k} \left(S_n \times H^{eff}_{k}\right) =
    \nabla_{S_k} \left(S_n \times\left(A_kS_k+B\right)\right) =\\=
    \nabla_{S_k} S_n \times \left( A_kS_k + B\right) + S_n\nabla_{S_k}
    \left(A_kS_k+B\right) =\\=
    \delta_{kn} \left(\colvec{1\\1\\1}\right) \times (A_kS_k+B) +
    S_n\times\nabla_{S_k}(A_kS_k+B) =\\=
    \delta_{kn} \left(\colvec{1\\1\\1}\right) \times H^{eff}_k + S_n A_k
\end{multline*}
\begin{multline*}
    \nabla_{S_k}\left(S_n\times\left(S_n\times H_k^{eff}\right)\right) =
    \nabla_{S_k}S_n\times\left(S_n\times H^{eff}_k\right) +
    S_n\times \nabla_{S_k}\left(S_n\times H^{eff}_k\right) =\\=
    \delta_{kn}\left(\colvec{1\\1\\1}\right)\times\left(S_n \times
    H^{eff}_k\right) +
    S_n\times \left(\delta_{kn}\left(\colvec{1\\1\\1}\right)\times H^{eff}_k +
    S_nA_k\right)
\end{multline*}
\begin{multline}
    \delta_{nk}\left(\colvec{1\\1\\1}\right) + h\sum_{i=1}^s \left( a_{ji}\cdot
     \gamma\left[
            \delta_{in} \left(\colvec{1\\1\\1}\right) \times H^{eff}_i + S_n
        A_i  \right] \right. -\\- \left. \gamma\lambda \left[
    S_n\times \left(\delta_{in}\left(\colvec{1\\1\\1}\right)\times H^{eff}_i +
S_nA_i\right) \right] \right) - \left(\colvec{1\\1\\1}\right) =
\nabla_{S_k}f
\end{multline}

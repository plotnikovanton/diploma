\section{Симплектический интегратор}

Для того чтобы сохранить сохранить энергию системы можно воспользоваться
симплектическим интегратором. Опираясь на работу \citet[стр.~3]{Markiewicz1999},
в общем виде итерационная система имеет вид:
\begin{figure}[h]
    \renewcommand{\arraystretch}{1.2}
    \centering
    \begin{tabular}{c|ccc}
        $c_1$    & $a_{11}$ & $\ldots$ & $a_{1s}$ \\
        $\vdots$ & $\vdots$ & $\ddots$ & $\vdots$ \\
        $c_s$    & $a_{11}$ & $\ldots$ & $a_{ss}$ \\ \hline
                 & $a_{11}$ & $\ldots$ & $a_{ss}$ \\
    \end{tabular}
    \caption{Таблица для общего вида}
\label{tab:tableau_basic}
\end{figure}

\begin{equation}\label{eq:base_form}
    \begin{gathered}
        S_{n+1} = S_n + h\sum_{j=1}^s b_j f(t_n+c_jh, \xi_j)
        \\
        \xi_j = y_n + h\sum_{i=1}^{s}a_{ji} f(t_n+c_jh, \xi_i)
    \end{gathered}
\end{equation}

В данной работе рассматривается симплектический интегратор Рунге-Кутта
второго порядка, он же метод Гаусса-Лежандра-Рунге-Кутта(далее ГЛРК). Для него
\fref{tab:tableau_basic} выглядит как \fref{tab:gauss-lagr}.

%TODO: increase row height
\begin{figure}[h]
    \centering
    \begin{tabular}{c|cc}
        $\frac12 - \frac{\sqrt3}6$ & $\frac14$                  & $\frac14 - \frac{\sqrt3}6$ \\
        $\frac12 + \frac{\sqrt3}6$ & $\frac14 + \frac{\sqrt3}6$ & $\frac14$ \\ \hline
                                   & $\frac12$                  & $\frac12$
    \end{tabular}
    \caption{Таблица для метода Гаусса-Лагранжа-Рунге-Кутта}
\label{tab:gauss-lagr}
\end{figure}

Для исследуемой модели $f(x)$ есть правая часть уравнения
Ландау-Лифшица \ref{REF_TO_EQUATION}.
Тогда итерационная схема~\ref{eq:base_form} для исследуемой модели
будет записана в виде:

\begin{gather}
    S_{n+1} = S_n + h\sum_{j=1}^s b_j \cdot
    \left(-\gamma S_n \times H^{eff}_n - \gamma\lambda S_n \times
    \left( S_n \times H^{eff}_n\right)\right)
    \label{eq:my_form}\\
    S_{k} = S_n + h\sum_{i=1}^s a_{ji}\cdot \left(
        -\gamma S_n \times H^{eff}_{k} - \gamma\lambda S_n \times
    \left(S_n \times H^{eff}_{k}\right)\right)
    \label{eq:rk_xi_equation}
\end{gather}

\begin{remark}
    Нужно отметить что в виде~\ref{eq:my_form} энергия сохраняться не будет из-за
    диссипации энергии. Поэтому далее, при проведении эксперимента, для наглядности того,
    что энергия сохраняется коэффициент диссипации $\lambda$ следует положить
    равным $0$.
\end{remark}

Для вычисления каждого следующего состояния системы необходимо решить
нелинейное уравнение~\ref{eq:rk_xi_equation}. Для этого можно воспользоваться
методом Ньютона.


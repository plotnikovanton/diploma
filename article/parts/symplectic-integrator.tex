\section{Симплектический интегратор}

Для того чтобы сохранить сохранить энергию системы можно воспользоваться
симплектическим интегратором. Опираясь на работу \citet[стр.~3]{Markiewicz1999},
в общем виде итерационная система имеет вид:
\begin{table}[h]\label{sym:tab:tableau_basic}
    \centering
    \begin{tabular}{c|ccc}
        $c_1$    & $a_{11}$ & $\ldots$ & $a_{1s}$ \\
        $\vdots$ & $\vdots$ & $\ddots$ & $\vdots$ \\
        $c_s$    & $a_{11}$ & $\ldots$ & $a_{ss}$ \\ \hline
                 & $a_{11}$ & $\ldots$ & $a_{ss}$ \\
    \end{tabular}
    \caption{Таблица для общего вида}
\end{table}

\begin{equation}\label{sym:eq:base_form}
    \begin{gathered}
        y_{n+1} = y_n + h\sum_{j=1}^s b_j f(t_n+c_jh, \xi_j)
        \\
        \xi_j = y_n + h\sum_{i=1}^{s}a_{ji} f(t_n+c_jh, \xi_i)
    \end{gathered}
\end{equation}

В данной работе рассматривается симплектический интегратор Рунге-Кутта
второго порядка, он же метод Гаусса-Лежандра-Рунге-Кутта. Для него таблица
\ref{sym:tab:tableau_basic} имеет вид:

\begin{table}
    \centering
    \begin{tabular}{c|cc}
        $\frac12 - \frac\sqrt36$ & $\frac14$ & \frac14 - frac\sqrt36 \\

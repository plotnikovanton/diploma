%\newcommand*{\heff}{\ensuremath{\mathbf{H}_{eff}}}
\newcommand*{\heff}{\ensuremath{\mathcal{H}}}
\newcommand*{\Sn}{\ensuremath{\mathbf{S}^{[n]}}}

\begin{itemize}
    \item $\Sn$ --- элемент вектора (столбец матрицы) $S$
        индексом $n$,
    \item $a \thicksim b$ --- обозначение наличия связи между атомами решетки с
        индексами $a$ и $b$,
    \item $\mathbf{E} $ --- полная энергия системы,
    \item $\mathbf{B} $ --- магнитное поле,
    \item $\mathbf{D}^{[n,m]} $ --- вектор Дзялошинского-Мория для пары
        соседних атомов $n$ и $m$,
    \item $\mathbf{J}^{[n,m]} $ --- коэффициент межатомного взаимодействия
        между атомами $n$ и $m$,
    \item $\mathbf{K} $ --- единичный вектор направления анизотропии,
    \item $\mathrm{K0} $ --- коэффициент анизотропии,
    \item $\delta_{kn} $ --- символ Кронекера,
    \item $\mathrm{Id_n} $ --- единичная матрица ранга $n$,
    \item $\dot a $ --- производная $a$ по времени,
    \item $\left<a | b\right> $ --- скалярное произведение векторов $a$ и $b$.
\end{itemize}
\newpage

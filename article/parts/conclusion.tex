
В результате работы представлен симплектический метод численного интегрирования
уравнения Ландау-Лифшица с помощью симплектического метода Рунге-Кутта и
продемонстрирована его эффективность.

Особенностью метода является отсутствие перехода из внешней трехмерной системы
координат к локальному двумерному базису, как часто делается в подобных
задачах. Такое решение требует дополнительных ресурсов для хранения данных о
третьей координате, но при этом операции над числами производятся более
простые, в реализованной программе присутствуют только операции сложения и
умножения матриц, что происходит значительно быстрее, чем операции, например,
$\sin$ или $\cos$, которые неизбежны при переходе к сферическим координатам.
